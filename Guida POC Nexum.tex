\documentclass[a4paper,12pt]{article}
\usepackage[utf8]{inputenc}
\usepackage[T1]{fontenc}
\usepackage[italian]{babel}
\usepackage[margin=2.5cm]{geometry}
\usepackage{graphicx}
\usepackage{grffile}
\usepackage{booktabs}
\usepackage{setspace}
\usepackage{titlesec}
\usepackage{float}
\usepackage{ifthen}
\usepackage{tcolorbox}
\usepackage{enumitem}
\usepackage{longtable}
\usepackage{array}
\usepackage{tabularx}
\usepackage{caption}
\usepackage{colortbl}
\usepackage{pgfplots}
\usepackage{xcolor}
\usepackage{listings}
\usepackage[colorlinks=true, linkcolor=black, urlcolor=blue, citecolor=black]{hyperref}
\usepackage{fancyhdr}
\usepackage{amssymb}
\usepackage{tocloft}

% Comando per la versione corrente
\newcommand{\CurrentVersion}{1.0.0}
\pgfplotsset{compat=1.16}

% Configurazione delle didascalie per tabelle
\captionsetup[table]{
  position=below,
  labelfont=it,
  textfont=it,
  justification=centering,
  singlelinecheck=false
}
\captionsetup[longtable]{
  position=below,
  labelfont=it,
  textfont=it,
  justification=centering,
  singlelinecheck=false
}

% Definizione colori
\definecolor{primaryblue}{RGB}{0,102,204}
\definecolor{secondaryblue}{RGB}{51,153,255}
\definecolor{lightgray}{RGB}{245,245,245}
\definecolor{darkgray}{RGB}{100,100,100}

% Formattazione titoli con colori
\titleformat{\section}
  {\Large\bfseries\color{primaryblue}}
  {\thesection}{1em}{}

\titleformat{\subsection}
  {\large\bfseries\color{secondaryblue}}
  {\thesubsection}{1em}{}

\titleformat{\subsubsection}
  {\normalsize\bfseries\color{secondaryblue}}
  {\thesubsubsection}{1em}{}

\setlength{\parskip}{4pt}
\setlength{\parindent}{0pt}
\setlength{\headheight}{14pt}

\setlist[itemize]{leftmargin=*,itemsep=3pt}
\setlist[enumerate]{leftmargin=*,itemsep=3pt}

% Configurazione per indice più dettagliato
\setcounter{tocdepth}{3}
\setcounter{secnumdepth}{3}

% Stili per i blocchi di codice
\lstset{
    basicstyle=\ttfamily\small,
    breaklines=true,
    frame=single,
    backgroundcolor=\color{gray!10},
    numbers=left,
    numberstyle=\tiny\color{gray},
    keywordstyle=\color{blue},
    commentstyle=\color{green!50!black},
    stringstyle=\color{orange},
    showstringspaces=false
}

% Intestazione e piè di pagina
\pagestyle{fancy}
\fancyhf{}
\fancyhead[L]{\small Guida POC Nexum}
\fancyhead[R]{\small \thepage}
\renewcommand{\headrulewidth}{0.4pt}

% Comando per registro versioni
\newcommand{\pediceG}{\textsuperscript{G}}

\begin{document}

% Prima pagina
\begin{center}
  \IfFileExists{../assets/Logo.jpg}{%
    \includegraphics[width=6cm,height=3cm,keepaspectratio]{../assets/Logo.jpg} \\[0.8cm]
  }{%
    \fbox{\parbox[c][2.5cm][c]{6cm}{\centering Logo non trovato\\(../assets/Logo.jpg)}}\\[0.5cm]
  }
  
  {\Large\bfseries\color{primaryblue} BugBusters}\\[0.5cm]

  {\Huge\bfseries\color{primaryblue} Come Usare il POC Nexum}\\[0.3cm]
  {\Large\color{secondaryblue} Guida all'installazione e utilizzo}\\[0.8cm]
\end{center}

\begin{center}
\begin{tcolorbox}[colback=lightgray,colframe=primaryblue,width=0.85\textwidth,arc=3mm,boxrule=0.5pt]
\begin{tabularx}{\linewidth}{@{}>{\raggedright\arraybackslash}p{3.5cm}>{\raggedright\arraybackslash}X@{}}
	{Stato} & In redazione \\
	{Responsabile} &  \\
	{Verificatore} &  \\
	{Redattori} & Alberto Autiero \\
	{Destinatari} & BugBusters, Eggon, Prof. Tullio Vardanega, Prof. Riccardo Cardin \\
\end{tabularx}
\end{tcolorbox}
\end{center}

% Pagina 2: Registro delle Modifiche
\newpage
\thispagestyle{empty}
\section*{\color{primaryblue}Registro delle Modifiche}

\arrayrulecolor{primaryblue}
{\footnotesize
\begin{tabularx}{\textwidth}{|>{\raggedright\arraybackslash}p{1.5cm}|>{\raggedright\arraybackslash}p{2cm}|X|>{\raggedright\arraybackslash}p{2cm}|>{\raggedright\arraybackslash}p{2cm}|>{\raggedright\arraybackslash}p{2cm}|}
\hline
\rowcolor{primaryblue!40}
\textbf{\color{white} Versione} & \textbf{\color{white} Data} & \textbf{\color{white} Descrizione} & \textbf{\color{white} Redatto} & \textbf{\color{white} Verificato} & \textbf{\color{white} Approvato} \\
\hline
0.0.1 & 29/12/2025 & Prima stesura della guida all'utilizzo del POC del documento & Alberto Autiero & - & - \\
\hline
\end{tabularx}
}

% Indice
\newpage
\pagestyle{fancy}
\tableofcontents
\newpage

% Sezione 1: Introduzione
\section{Introduzione}
\label{sec:introduzione}
Questa guida spiega in dettaglio come configurare, avviare e utilizzare il Proof of Concept (POC) Nexum. Seguire attentamente le istruzioni per garantire il corretto funzionamento di tutti i componenti.

% Sezione 2: Prerequisiti
\section{Prerequisiti}
\label{sec:prerequisiti}
Prima di iniziare, assicurarsi di avere installato:

\begin{itemize}[leftmargin=*]
    \item \textbf{Ruby} (versione 3.0 o superiore)
    \item \textbf{PostgreSQL} (versione 12 o superiore)
    \item \textbf{Node.js} (versione 16 o superiore, per il frontend Angular)
    \item \textbf{Credenziali AWS} con accesso a Bedrock
    \item \textbf{PowerShell} (già incluso in Windows)
\end{itemize}

% Sezione 3: Configurazione Iniziale
\section{Configurazione Iniziale}
\label{sec:configurazione}

\subsection{Configurare PostgreSQL}
\label{subsec:postgresql}
Il POC richiede PostgreSQL in esecuzione con le credenziali corrette.

\subsubsection{Modificare la password nel file di configurazione}
\label{subsubsec:password-config}
Aprire il file \texttt{backend/config/database.yml} e modificare la password PostgreSQL in tutte le sezioni (development, test, production):

\begin{lstlisting}[caption=Configurazione database.yml]
development:
  <<: *default
  database: nexum_poc_development
  username: postgres
  password: "TUA_PASSWORD_QUI"  # <-- Modifica questa riga
  host: 127.0.0.1
  port: 5432
\end{lstlisting}

\textbf{Nota}: Sostituire \texttt{"TUA\_PASSWORD\_QUI"} con la password dell'utente PostgreSQL.

\subsubsection{Verificare che PostgreSQL sia in esecuzione}
\label{subsubsec:verifica-postgres}
Verificare che PostgreSQL sia attivo eseguendo:

\begin{lstlisting}[language=bash, caption=Verifica PostgreSQL]
Test-NetConnection -ComputerName localhost -Port 5432
\end{lstlisting}

Se la porta non risponde, avviare il servizio PostgreSQL da "Servizi" di Windows.

\subsection{Configurare le Credenziali AWS}
\label{subsec:aws}
Il POC necessita delle credenziali AWS per accedere a Bedrock. Creare un file \texttt{.env} nella cartella \texttt{backend}.

\subsubsection{Creare il file .env}
\label{subsubsec:crea-env}
Creare un file chiamato \texttt{.env} nella cartella \texttt{backend/} con il seguente contenuto:

\begin{lstlisting}[caption=File .env di esempio]
AWS_ACCESS_KEY_ID=la_tua_access_key
AWS_SECRET_ACCESS_KEY=la_tua_secret_key
AWS_SESSION_TOKEN=la_tua_session_token
AWS_REGION=us-east-1
\end{lstlisting}

\textbf{Dove trovare le credenziali AWS:}

\begin{enumerate}
    \item Accedere al portale AWS IAM Identity Center (solitamente \url{https://eggon.awsapps.com/start/})
    \item Selezionare il ruolo "Bedrock-Bugbuster"
    \item Copiare le credenziali temporanee (Access Key ID, Secret Access Key, Session Token)
    \item Incollare i valori nel file \texttt{.env}
\end{enumerate}

\textbf{Importante:}
\begin{itemize}
    \item Il file \texttt{.env} è già ignorato da git (non verrà committato)
    \item Le credenziali AWS scadono dopo alcune ore
    \item Quando le credenziali scadono, crearne di nuove dal portale AWS
\end{itemize}

% Sezione 4: Avvio del POC
\section{Avvio del POC}
\label{sec:avvio}

\subsection{Script PowerShell avvia-poc.ps1}
\label{subsec:script-avvio}
Il modo più semplice per avviare tutto:

\begin{lstlisting}[language=bash, caption=Script combinato]
.\avvia-poc.ps1
\end{lstlisting}

Questo script:
\begin{itemize}
    \item Apre una nuova finestra PowerShell per il Backend
    \item Apre una nuova finestra PowerShell per il Frontend
    \item Mostra un messaggio con tutti gli URL disponibili
\end{itemize}

\textbf{Per fermare i servizi}: Premere CTRL+C

% Sezione 5: URL Disponibili
\section{URL Disponibili}
\label{sec:url}
Una volta avviati i servizi, avrai accesso a:

\subsection{Applicazioni Principali}
\label{subsec:app-principali}
\begin{itemize}
    \item \textbf{Frontend Angular}: \url{http://localhost:4200}
    \item \textbf{Backend Rails API}: \url{http://localhost:3000}
\end{itemize}

\subsection{Pagine di Test HTML}
\label{subsec:pagine-test}
\begin{itemize}
    \item \textbf{Generazione Testo}: \url{http://localhost:3000/tester.html}
    \item \textbf{Analisi Documenti}: \url{http://localhost:3000/documentTester.html}
\end{itemize}

% Sezione 6: Come Testare il POC
\section{Come Testare il POC}
\label{sec:test}

\subsection{Test 1: Analisi Documenti}
\label{subsec:test-analisi}
\begin{enumerate}
    \item Aprire \url{http://localhost:3000/documentTester.html}
    \item Assicurarsi che "Company ID" sia impostato a \texttt{1}
    \item Cliccare su "Scegli file" e selezionare \texttt{documento-test.pdf} presente nella cartella backend 
    \item Cliccare su "Carica e Analizza"
    \item Attendere che l'AI analizzi il documento
    \item Visualizzare i risultati estratti
\end{enumerate}

\textbf{Creare un PDF di test:}
Se non hai il file \texttt{documento-test.pdf}, puoi crearne uno eseguendo:

\begin{lstlisting}[language=bash]
ruby crea-pdf-test.rb
\end{lstlisting}

\subsection{Test 2: Generazione Testo con Conversazioni}
\label{subsec:test-generazione}
\begin{enumerate}
    \item Aprire \url{http://localhost:3000/tester.html}
    \item Impostare "Company ID" a \texttt{1}
    \item Cliccare su "Carica toni" per vedere i toni disponibili
    \item Selezionare un tono dal menu a tendina
    \item Scrivere un prompt (es: "Scrivi un'email di presentazione")
    \item Cliccare su "Genera"
    \item Visualizzare la risposta dell'AI
\end{enumerate}

% Sezione 7: Script PowerShell Disponibili
\section{Script PowerShell Disponibili}
\label{sec:script}
Il progetto include diversi script PowerShell utili:

\begin{table}[h]
\centering
\begin{tabularx}{\textwidth}{lX}
\toprule
\textbf{Script} & \textbf{Descrizione} \\
\midrule
\texttt{avvia-poc.ps1} & Avvia sia Backend che Frontend (apre due finestre) \\
\texttt{avvia-backend.ps1} & Avvia solo il Backend Rails \\
\texttt{avvia-frontend.ps1} & Avvia solo il Frontend Angular \\
\texttt{verifica-postgres.ps1} & Verifica la connessione a PostgreSQL (chiede password) \\
\texttt{crea-pdf-test.rb} & Crea un PDF di test per l'API di analisi \\
\bottomrule
\end{tabularx}
\caption{Script disponibili nel progetto}
\label{tab:script}
\end{table}

% Sezione 8: Troubleshooting
\section{Troubleshooting}
\label{sec:troubleshooting}

\subsection{Errore: "company\_id mancante"}
\label{subsec:errore-company}
\begin{itemize}
    \item Assicurati di includere \texttt{company\_id=1} nella query string o nel form
    \item Le pagine HTML di test ora includono un campo per il Company ID
\end{itemize}

\subsection{Errore: "Database non inizializzato"}
\label{subsec:errore-database}
\begin{itemize}
    \item Esegui manualmente: \texttt{cd backend \&\& bundle exec rails db:create db:migrate db:seed}
\end{itemize}

\subsection{Errore: "Password PostgreSQL errata"}
\label{subsec:errore-password}
\begin{itemize}
    \item Verifica che la password in \texttt{backend/config/database.yml} corrisponda alla password del tuo utente PostgreSQL
    \item Usa lo script \texttt{verifica-postgres.ps1} per testare la connessione
\end{itemize}

\subsection{Errore: "AWS credentials expired"}
\label{subsec:errore-aws}
\begin{itemize}
    \item Le credenziali AWS scadono dopo alcune ore
    \item Ottieni nuove credenziali dal portale AWS IAM Identity Center
    \item Aggiorna il file \texttt{backend/.env} con le nuove credenziali
\end{itemize}

\subsection{Errore: "Could not find gem"}
\label{subsec:errore-gem}
\begin{itemize}
    \item Esegui: \texttt{cd backend \&\& bundle install}
\end{itemize}

\subsection{Errore: "Port already in use"}
\label{subsec:errore-porta}
\begin{itemize}
    \item La porta 3000 (backend) o 4200 (frontend) è già in uso
    \item Chiudi l'applicazione che usa quella porta o modifica la porta nel file di configurazione
\end{itemize}

\subsection{Il server non si avvia}
\label{subsec:errore-server}
\begin{itemize}
    \item Verifica che PostgreSQL sia in esecuzione
    \item Controlla che le variabili AWS siano configurate nel file \texttt{.env}
    \item Controlla i log in \texttt{backend/log/development.log}
\end{itemize}

\subsection{Warning VIPS}
\label{subsec:warning-vips}
\begin{itemize}
    \item I warning VIPS (riguardanti moduli opzionali per immagini) possono essere ignorati - non bloccano il funzionamento dell'applicazione.
\end{itemize}

% Sezione 9: Note Importanti
\section{Note Importanti}
\label{sec:note}
\begin{itemize}
    \item \textbf{Database}: Il seed iniziale crea una Company con ID=1, alcuni Tone di esempio e una conversazione di esempio
    \item \textbf{Company ID}: La maggior parte delle API richiede \texttt{company\_id=1} (la company di default)
    \item \textbf{Storage}: I file caricati vengono salvati localmente in \texttt{backend/storage/}
    \item \textbf{Log}: I log del backend sono disponibili in \texttt{backend/log/development.log}
    \item \textbf{Hot Reload}:
    \begin{itemize}
        \item Rails: Si ricarica automaticamente quando modifichi i file Ruby
        \item Angular: Si ricompila automaticamente quando modifichi i file TypeScript/HTML
    \end{itemize}
\end{itemize}

% Sezione 10: Riavvio Dopo Modifiche
\section{Riavvio Dopo Modifiche}
\label{sec:riavvio}
Dopo aver modificato:

\begin{itemize}
    \item \textbf{File Ruby (modelli, controller, ecc.)}: Rails si ricarica automaticamente
    \item \textbf{File di configurazione Rails}: Riavviare il POC (Ctrl+C e rieseguire \texttt{.\textbackslash avvia-poc.ps1})
    \item \textbf{File Angular}: Angular si ricompila automaticamente
    \item \textbf{File .env}: Riavviare il POC (Ctrl+C e rieseguire \texttt{.\textbackslash avvia-poc.ps1)}
\end{itemize}

% Sezione 11: Risorse Aggiuntive
\section{Risorse Aggiuntive}
\label{sec:risorse}
\begin{itemize}
    \item \textbf{Documentazione Backend}: Vedere \texttt{backend/docs/SETUP\_E\_DOCUMENTAZIONE.txt}
    \item \textbf{Gemfile}: Lista delle dipendenze Ruby in \texttt{backend/Gemfile}
    \item \textbf{package.json}: Lista delle dipendenze Node.js in \texttt{frontend/package.json}
\end{itemize}

\end{document}