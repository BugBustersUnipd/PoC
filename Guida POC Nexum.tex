\documentclass[a4paper,12pt]{article}
\usepackage[utf8]{inputenc}
\usepackage[T1]{fontenc}
\usepackage[italian]{babel}
\usepackage[margin=2.5cm]{geometry}
\usepackage{graphicx}
\usepackage{grffile}
\usepackage{booktabs}
\usepackage{setspace}
\usepackage{titlesec}
\usepackage{float}
\usepackage{ifthen}
\usepackage{tcolorbox}
\usepackage{enumitem}
\usepackage{longtable}
\usepackage{array}
\usepackage{tabularx}
\usepackage{caption}
\usepackage{colortbl}
\usepackage{pgfplots}
\usepackage{xcolor}
\usepackage{listings}
\usepackage[colorlinks=true, linkcolor=black, urlcolor=blue, citecolor=black]{hyperref}
\usepackage{fancyhdr}
\usepackage{amssymb}
\usepackage{tocloft}

% Comando per la versione corrente
\newcommand{\CurrentVersion}{1.0.0}
\pgfplotsset{compat=1.16}

% Configurazione delle didascalie per tabelle
\captionsetup[table]{
  position=below,
  labelfont=it,
  textfont=it,
  justification=centering,
  singlelinecheck=false
}
\captionsetup[longtable]{
  position=below,
  labelfont=it,
  textfont=it,
  justification=centering,
  singlelinecheck=false
}

% Definizione colori
\definecolor{primaryblue}{RGB}{0,102,204}
\definecolor{secondaryblue}{RGB}{51,153,255}
\definecolor{lightgray}{RGB}{245,245,245}
\definecolor{darkgray}{RGB}{100,100,100}

% Formattazione titoli con colori
\titleformat{\section}
  {\Large\bfseries\color{primaryblue}}
  {\thesection}{1em}{}

\titleformat{\subsection}
  {\large\bfseries\color{secondaryblue}}
  {\thesubsection}{1em}{}

\titleformat{\subsubsection}
  {\normalsize\bfseries\color{secondaryblue}}
  {\thesubsubsection}{1em}{}

\setlength{\parskip}{4pt}
\setlength{\parindent}{0pt}
\setlength{\headheight}{14pt}

\setlist[itemize]{leftmargin=*,itemsep=3pt}
\setlist[enumerate]{leftmargin=*,itemsep=3pt}

% Configurazione per indice più dettagliato
\setcounter{tocdepth}{3}
\setcounter{secnumdepth}{3}

% Stili per i blocchi di codice
\lstset{
    basicstyle=\ttfamily\small,
    breaklines=true,
    frame=single,
    backgroundcolor=\color{gray!10},
    numbers=none,
    numberstyle=\tiny\color{gray},
    keywordstyle=\color{blue},
    commentstyle=\color{green!50!black},
    stringstyle=\color{orange},
    showstringspaces=false
}

% Intestazione e piè di pagina
\pagestyle{fancy}
\fancyhf{}
\fancyhead[L]{}
\fancyhead[R]{}
\fancyfoot[L]{}
\fancyfoot[R]{\small \thepage}
\renewcommand{\headrulewidth}{0pt}
\renewcommand{\footrulewidth}{0pt}

% Comando per registro versioni
\newcommand{\pediceG}{\textsuperscript{G}}

\begin{document}

% Prima pagina
\begin{center}
  \IfFileExists{assets/Logo.jpg}{%
    \includegraphics[width=6cm,height=3cm,keepaspectratio]{assets/Logo.jpg} \\[0.8cm]
  }{%
    \fbox{\parbox[c][2.5cm][c]{6cm}{\centering Logo non trovato\\(assets/Logo.jpg)}}\\[0.5cm]
  }
  
  {\Large\bfseries\color{primaryblue} BugBusters}\\[0.5cm]

  {\Huge\bfseries\color{primaryblue} Come Usare il POC Nexum}\\[0.3cm]
  {\Large\color{secondaryblue} Guida all'installazione e utilizzo}\\[0.8cm]
\end{center}

\begin{center}
\begin{tcolorbox}[colback=lightgray,colframe=primaryblue,width=0.85\textwidth,arc=3mm,boxrule=0.5pt]
\begin{tabularx}{\linewidth}{@{}>{\raggedright\arraybackslash}p{3.5cm}>{\raggedright\arraybackslash}X@{}}
	{Stato} & In redazione \\
	{Verificatore} &  \\
	{Redattori} & Alberto Autiero \\
	{Destinatari} & BugBusters, Eggon, Prof. Tullio Vardanega, Prof. Riccardo Cardin \\
\end{tabularx}
\end{tcolorbox}
\end{center}

% Pagina 2: Registro delle Modifiche
\newpage
\thispagestyle{empty}
\section*{\color{primaryblue}Registro delle Modifiche}

\arrayrulecolor{primaryblue}
{\footnotesize
\begin{tabularx}{\textwidth}{|>{\raggedright\arraybackslash}p{1.5cm}|>{\raggedright\arraybackslash}p{2cm}|X|>{\raggedright\arraybackslash}p{2cm}|>{\raggedright\arraybackslash}p{2cm}|>{\raggedright\arraybackslash}p{2cm}|}
\hline
\rowcolor{primaryblue!40}
\textbf{\color{white} Versione} & \textbf{\color{white} Data} & \textbf{\color{white} Descrizione} & \textbf{\color{white} Redatto} & \textbf{\color{white} Verificato} & \textbf{\color{white} Approvato} \\
0.0.3 & 05/05/2026 & Aggiunta sezione installazione su MACOS e Linux & Marco Favero & - & - \\
\hline
0.0.2 & 31/12/2025 & Migliorie alla guida & Alberto Autiero & - & - \\
\hline
0.0.1 & 29/12/2025 & Prima stesura della guida all'utilizzo del POC Nexum & Alberto Autiero & - & - \\
\hline
\end{tabularx}
}

% Indice
\newpage
\pagestyle{fancy}
\tableofcontents
\newpage

% Sezione 1: Introduzione
\section{Introduzione}
\label{sec:introduzione}
Questa guida spiega in dettaglio come configurare, avviare e utilizzare il Proof of Concept (POC) Nexum. Seguire attentamente le istruzioni per garantire il corretto funzionamento di tutti i componenti.

% Sezione 3: Installazione - Windows
\section{Installazione - Windows}
\label{sec:installazione-windows}

\subsection{Ruby}
\label{subsec:win-ruby}

\begin{enumerate}
    \item Visitare \url{https://rubyinstaller.org/downloads/}
    \item Scaricare \textbf{Ruby+Devkit} versione x64 (64-bit)
    \item Eseguire il file di installazione
    \item Durante l'installazione, selezionare tutte le opzioni predefinite
    \item \textbf{Importante}: Installare MSYS2 e le componenti di sviluppo (DevKit)
    \item Verificare l'installazione:
    \begin{lstlisting}[language=bash, caption=Verifica Ruby]
ruby --version
    \end{lstlisting}
    \item Installare Bundler:
    \begin{lstlisting}[language=bash, caption=Installazione Bundler]
gem install bundler
    \end{lstlisting}
\end{enumerate}

\subsection{PostgreSQL}
\label{subsec:win-postgresql}

\begin{enumerate}
    \item Visitare \url{https://www.postgresql.org/download/windows/}
    \item Scaricare PostgreSQL Installer per Windows
    \item Eseguire il file di installazione
    \item Durante l'installazione:
    \begin{itemize}
        \item Scegliere le componenti predefinite
        \item \textbf{Importante}: Impostare una password per l'utente \texttt{postgres} (non dimenticarla!)
        \item Lasciare la porta predefinita (5432)
    \end{itemize}
    \item Configurare l'avvio automatico (opzionale):
    \begin{lstlisting}[language=bash, caption=Avvio automatico PostgreSQL su Windows]
Set-Service postgresql-x64-18 -StartupType Automatic
    \end{lstlisting}
    \textbf{Nota}: Sostituire il numero di versione se diverso.
\end{enumerate}

\subsection{Node.js}
\label{subsec:win-nodejs}

\begin{enumerate}
    \item Aprire PowerShell e eseguire:
    \begin{lstlisting}[language=bash, caption=Installazione Node.js con winget]
winget install OpenJS.NodeJS.LTS
    \end{lstlisting}
    \item Verificare l'installazione:
    \begin{lstlisting}[language=bash, caption=Verifica Node.js]
node --version
npm --version
    \end{lstlisting}
    \item Installare le dipendenze del progetto (nella cartella frontend):
    \begin{lstlisting}[language=bash, caption=Dipendenze frontend su Windows]
cd frontend
cmd /c "npm install"
    \end{lstlisting}
\end{enumerate}

\subsection{Configurare PowerShell}
\label{subsec:win-powershell}

\begin{lstlisting}[language=bash, caption=Policy di esecuzione PowerShell]
Set-ExecutionPolicy -ExecutionPolicy RemoteSigned -Scope CurrentUser
\end{lstlisting}

\textbf{Nota}: Confermare con \texttt{Y} o \texttt{S}.

% Sezione 3b: Installazione - Linux e macOS
\section{Installazione - Linux e macOS}
\label{sec:installazione-linux-macos}

\subsection{Ruby}
\label{subsec:lm-ruby}

\begin{itemize}
    \item \textbf{macOS}: \texttt{brew install ruby}
    \item \textbf{Linux (Ubuntu/Debian)}: \texttt{sudo apt-get install ruby-full build-essential}
    \item \textbf{Linux (Fedora/RHEL)}: \texttt{sudo dnf install ruby ruby-devel}
\end{itemize}

Dopo l'installazione:
\begin{lstlisting}[language=bash, caption=Verificare e completare Ruby]
ruby --version
gem install bundler
\end{lstlisting}

\subsection{PostgreSQL}
\label{subsec:lm-postgresql}

\textbf{macOS}:
\begin{lstlisting}[language=bash, caption=PostgreSQL su macOS]
brew install postgresql@18
brew services start postgresql@18
\end{lstlisting}

\textbf{Linux (Ubuntu/Debian)}:
\begin{lstlisting}[language=bash, caption=PostgreSQL su Ubuntu/Debian]
sudo apt-get install postgresql postgresql-contrib
sudo systemctl start postgresql
sudo systemctl enable postgresql
\end{lstlisting}

\textbf{Linux (Fedora/RHEL)}:
\begin{lstlisting}[language=bash, caption=PostgreSQL su Fedora/RHEL]
sudo dnf install postgresql postgresql-server
sudo systemctl start postgresql
sudo systemctl enable postgresql
\end{lstlisting}

\subsection{Node.js}
\label{subsec:lm-nodejs}

\textbf{macOS}:
\begin{lstlisting}[language=bash, caption=Node.js su macOS]
brew install node
\end{lstlisting}

\textbf{Linux}: Seguire \url{https://nodejs.org/en/download/package-manager}

Installare le dipendenze del progetto:
\begin{lstlisting}[language=bash, caption=Dipendenze frontend su Linux/macOS]
cd frontend
npm install
\end{lstlisting}

% Sezione 4: Configurazione Iniziale
\section{Configurazione Iniziale}
\label{sec:configurazione}

\subsection{Configurare le Variabili d'Ambiente}
\label{subsec:env-config}

La configurazione del progetto avviene tramite il file \texttt{backend/.env}. Una copia di esempio è fornita in \texttt{backend/.env.example}.

\subsubsection{Creare il file .env}
\label{subsubsec:create-env}

Copiare \texttt{.env.example} e rinominarlo in \texttt{.env}:

\textbf{Windows (PowerShell)}:
\begin{lstlisting}[language=bash, caption=Creare .env su Windows]
cd backend
Copy-Item .env.example .env
\end{lstlisting}

\textbf{Linux/macOS}:
\begin{lstlisting}[language=bash, caption=Creare .env su Linux/macOS]
cd backend
cp .env.example .env
\end{lstlisting}

\subsubsection{Completare le Credenziali AWS}
\label{subsubsec:aws-credentials}

Aprire il file \texttt{backend/.env} e compilare le credenziali AWS:

\begin{lstlisting}[caption=Sezione AWS in .env]
AWS_ACCESS_KEY_ID=la_tua_access_key
AWS_SECRET_ACCESS_KEY=la_tua_secret_key
AWS_SESSION_TOKEN=la_tua_session_token
\end{lstlisting}

\textbf{Dove trovarle}: Accedere a \url{https://eggon.awsapps.com/start/}, selezionare il ruolo ``Bedrock-Bugbuster'' e copiare le credenziali temporanee.

\subsubsection{Configurare il Database}
\label{subsubsec:db-password}

Nel file \texttt{backend/.env}, impostare la password PostgreSQL:

\begin{lstlisting}[caption=Database in .env]
DB_HOST=localhost
DB_PORT=5432
DB_USER=postgres
DB_PASSWORD=la_tua_password_postgresql
DB_NAME_DEVELOPMENT=nexum_poc_development
DB_NAME_TEST=nexum_poc_test
\end{lstlisting}

\textbf{Nota}: \texttt{DB\_PASSWORD} deve essere la password dell'utente \texttt{postgres} impostata durante l'installazione di PostgreSQL.

\subsection{Inizializzare il Database}
\label{subsec:init-db}

\textbf{Windows}:
\begin{lstlisting}[language=bash, caption=Inizializzare DB su Windows]
# Avviare PostgreSQL, alternativamente ad aprire pgAdmin
Start-Service postgresql-x64-18 

# Dalla root del progetto, eseguire
cd backend
bundle exec rails db:create db:migrate db:seed
\end{lstlisting}

\textbf{Linux/macOS}:
\begin{lstlisting}[language=bash, caption=Inizializzare DB su Linux/macOS]
cd backend
bundle exec rails db:create db:migrate db:seed
\end{lstlisting}

% Sezione 5: Avvio del POC
\section{Avvio del POC}
\label{sec:avvio}

\subsection{Windows - Con Script PowerShell}
\label{subsec:avvio-windows}

Eseguire il comando nella root del progetto:

\begin{lstlisting}[language=bash, caption=Avviare il POC su Windows]
.\avvia-poc.ps1
\end{lstlisting}

Questo script apre due finestre PowerShell: una per il Backend e una per il Frontend.

\subsection{Linux/macOS - Manuale}
\label{subsec:avvio-linux}

\subsubsection{Terminale 1: Backend}
\label{subsubsec:avvio-backend}

\begin{lstlisting}[language=bash, caption=Backend su Linux/macOS]
cd backend
bundle exec rails server
\end{lstlisting}

\subsubsection{Terminale 2: Frontend}
\label{subsubsec:avvio-frontend}

\begin{lstlisting}[language=bash, caption=Frontend su Linux/macOS]
cd frontend
npm start
\end{lstlisting}

% Sezione 6: URL Disponibili
\section{URL Disponibili}
\label{sec:url}
Una volta avviati i servizi, avrai accesso a:

\subsection{Applicazioni Principali}
\label{subsec:app-principali}
\begin{itemize}
    \item \textbf{Frontend Angular}: \url{http://localhost:4200}
    \item \textbf{Backend Rails API}: \url{http://localhost:3000}
\end{itemize}

% Sezione 7: Script PowerShell Disponibili
\section{Script PowerShell Disponibili}
\label{sec:script}
Il progetto include diversi script PowerShell utili:

\begin{table}[h]
\centering
\begin{tabularx}{\textwidth}{lX}
\toprule
\textbf{Script} & \textbf{Descrizione} \\
\midrule
\texttt{avvia-poc.ps1} & Avvia sia Backend che Frontend (apre due finestre) \\
\texttt{avvia-backend.ps1} & Avvia solo il Backend \\
\texttt{avvia-frontend.ps1} & Avvia solo il Frontend \\
\texttt{verifica-postgres.ps1} & Verifica la connessione a PostgreSQL (chiede password) \\
\bottomrule
\end{tabularx}
\caption{Script disponibili nel progetto}
\label{tab:script}
\end{table}

% Sezione 8: Troubleshooting
\section{Troubleshooting}
\label{sec:troubleshooting}

\subsection{Errore: "Problemi di permessi in fase di configurazione"}
\label{subsec:errore-permessi}
\begin{itemize}
    \item Esegui PowerShell/cmd come amministratore o usa \texttt{sudo} su Linux/macOS.
\end{itemize}

\subsection{Errore: "Database non inizializzato"}
\label{subsec:errore-database}
\begin{itemize}
    \item Esegui manualmente: \texttt{cd backend \&\& bundle exec rails db:create db:migrate db:seed}
\end{itemize}

\subsection{Errore: "Password PostgreSQL errata"}
\label{subsec:errore-password}
\begin{itemize}
    \item Verifica che la password in \texttt{backend/.env} corrisponda alla password dell'utente \texttt{postgres} impostata durante l'installazione
    \item Usa lo script \texttt{verifica-postgres.ps1} (su Windows) per testare la connessione
\end{itemize}

\subsection{Errore: "AWS credentials expired"}
\label{subsec:errore-aws}
\begin{itemize}
    \item Le credenziali AWS scadono dopo alcune ore
    \item Ottieni nuove credenziali dal portale AWS IAM Identity Center
    \item Aggiorna il file \texttt{backend/.env} con le nuove credenziali
\end{itemize}

\subsection{Errore: "Could not find gem"}
\label{subsec:errore-gem}
\begin{itemize}
    \item Esegui: \texttt{cd backend \&\& bundle install}
\end{itemize}

\subsection{Errore: "Port already in use"}
\label{subsec:errore-porta}
\begin{itemize}
    \item La porta 3000 (backend) o 4200 (frontend) è già in uso
    \item Chiudi l'applicazione che usa quella porta o modifica la porta nel file di configurazione
\end{itemize}

\subsection{Il server non si avvia}
\label{subsec:errore-server}
\begin{itemize}
    \item Verifica che PostgreSQL sia in esecuzione
    \item Controlla che le variabili AWS siano configurate nel file \texttt{.env}
    \item Controlla i log in \texttt{backend/log/development.log}
\end{itemize}

\subsection{Warning VIPS}
\label{subsec:warning-vips}
\begin{itemize}
    \item I warning VIPS (riguardanti moduli opzionali per immagini) possono essere ignorati - non bloccano il funzionamento dell'applicazione.
\end{itemize}

\subsection{Errore: "Script execution is disabled on this system"}
\label{subsec:errore-powershell-policy}
Se ricevi un errore quando provi a eseguire gli script PowerShell:

\textbf{Soluzione 1: Cambiare la policy di esecuzione (permanente)}
\label{subsubsec:soluzione-powershell-1}
Per l'utente corrente:

\begin{lstlisting}[language=bash, caption=Cambiare policy di esecuzione PowerShell]
Set-ExecutionPolicy -ExecutionPolicy RemoteSigned -Scope CurrentUser
\end{lstlisting}

Questo comando permette l'esecuzione di script locali e richiede la firma solo per script scaricati da Internet.

\textbf{Nota}: Potrebbe essere richiesto di confermare con \texttt{S} o \texttt{Y}.

% Sezione 9: Note Importanti
\section{Note Importanti}
\label{sec:note}
\begin{itemize}
    \item \textbf{Database}: Il seed iniziale crea una Company con ID=1, alcuni Tone di esempio e una conversazione di esempio
    \item \textbf{Company ID}: La maggior parte delle API richiede \texttt{company\_id=1} (la company di default)
    \item \textbf{Storage}: I file caricati vengono salvati localmente in \texttt{backend/storage/}
    \item \textbf{Log}: I log del backend sono disponibili in \texttt{backend/log/development.log}
    \item \textbf{Hot Reload}:
    \begin{itemize}
        \item Rails: Si ricarica automaticamente quando modifichi i file Ruby
        \item Angular: Si ricompila automaticamente quando modifichi i file TypeScript/HTML
    \end{itemize}
\end{itemize}

% Sezione 10: Riavvio Dopo Modifiche
\section{Riavvio Dopo Modifiche}
\label{sec:riavvio}
Dopo aver modificato:

\begin{itemize}
    \item \textbf{File Ruby (modelli, controller, ecc.)}: Rails si ricarica automaticamente
    \item \textbf{File di configurazione Rails}: Riavviare il POC (Ctrl+C e rieseguire lo script di avvio)
    \item \textbf{File Angular}: Angular si ricompila automaticamente
    \item \textbf{File .env}: Riavviare il POC (Ctrl+C e rieseguire lo script di avvio)
\end{itemize}

% Sezione 11: Risorse Aggiuntive
\section{Risorse Aggiuntive}
\label{sec:risorse}
\begin{itemize}
    \item \textbf{Gemfile}: Lista delle dipendenze Ruby in \texttt{backend/Gemfile}
    \item \textbf{package.json}: Lista delle dipendenze Node.js in \texttt{frontend/package.json}
\end{itemize}

\end{document}